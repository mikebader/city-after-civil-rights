\documentclass[10pt]{homework}

\title{Class Policies \& Resources}
\duedate[]{\ }
\coursetitle{The City After Civil Rights}
\coursenumber{SOCY 368}
\author{Prof. Michael Bader}

\begin{document}
\maketitle 

\vspace{-3em}
\section{Policies} 

\subsection{Respect in (and out of) the Classroom}

The \textbf{single MOST IMPORTANT} policy is the following: \emph{treat everyone with respect, as they ask to be treated}.\\[0.25\baselineskip]

Every one of us brings something new, interesting, and engaging to the topics that we will discuss. This does not mean you cannot disagree and express disagreements; however, we can all expect that each other does so with the respect we desire to be treated with ourselves. Not everyone expects or wants the same treatment. You may start from treating others how you would want to be treated, but if they indicate that they desire something else, please listen. 

If you have concerns, please raise them with me. 

\subsection{Incomplete Grades}

The ongoing pandemic means that we will all need to be flexible with one another. While being flexible, I also want to state, unequivocally, that taking an incomplete diminishes the value of what you learn in class. I gear the assignments to match the material. It becomes very difficult to catch up if you get too far behind. For that reason, I will consider incomplete grades only in extreme extenuating circumstances and would prefer, in contrast, to work with you during the semester to adjust the material so that you may finish the course in December.

\subsection{Abiding by the Student Code of Conduct}

The central commitment of American University is the development of thoughtful, responsible human beings in the context of a challenging yet supportive academic community. The \href{https://www.american.edu/policies/students/student-conduct-code.cfm}{Student Code of Conduct} is designed to benefit the American University community and to assist in forming the highest standards of ethics and morals among its members. By registering for this course, students have acknowledged their awareness of the Student Code of Conduct and they are obliged to become familiar with their rights and responsibilities as defined by the Code.

\href{https://www.american.edu/policies/au-community/upload/disrimination-and-sexual-harassment-policy-09-05-19-final.pdf}{American University expressly prohibits any form of discrimination and discriminatory harassment} including sexual harassment, dating and domestic violence, sexual assault, and stalking. As a faculty member, \emph{I am required to report} discriminatory or harassing conduct to the university if I witness it or become aware of it--regardless of the location of the incident. The same is true if I believe that you are in danger of harm or in danger of harming someone else. There are four confidential resources on campus if you wish to speak to someone: \href{https://www.american.edu/ocl/counseling/}{Counseling Center}, \href{https://www.american.edu/ocl/promote-health/oasis.cfm}{victim advocates in OASIS}, medical providers in the \href{https://www.american.edu/ocl/healthcenter/}{Student Health Center}, and ordained clergy in the \href{https://www.american.edu/ocl/kay/}{Kay Spiritual Life Center}. If you experience any of the above, you have the option of filing a report (details about how can be found on the \href{https://www.american.edu/equity-titleix/help.cfm}{Office of Equity \& Title IX}). For more information, including a list of supportive resources see the Office's \href{https://www.american.edu/equity-titleix/support.cfm}{support guide}.


\subsection{Academic Integrity}

I expect that you will follow university guidelines regarding academic integrity on all assignments (graded or not). I expect that you will provide appropriate credit through citation and acknowledgement of others' ideas while you take this class. I am required to report all suspected violations of the university's code of conduct to the Dean and will do so when I suspect plagiarism or academic dishonesty.

Standards of academic conduct are set forth in the university’s Academic Integrity Code. By registering for this course, you have acknowledged your awareness of the \href{https://www.american.edu/academics/integrity/code.cfm}{Academic Integrity Code} and you must become familiar with your rights and responsibilities as defined by the Code. 

\subsection{Religious Observances}

Students will be provided the opportunity to make up any examination, study, or work requirements that may be missed due to a religious observance, provided they notify their instructors before the end of the second week of classes. Please send this notification through email to the professor. For additional information, see American University’s \href{http://www.american.edu/ocl/kay/Major-Religious-Holy-Days.cfm}{religious observances policy}.


\section{Resources}

\subsection{Academic Support and Access Center (ASAC) MGC 243,
202-885-3360.} All students may take advantage of the Academic
Support and Access Center (ASAC) for individual academic skills,
counseling, workshops, tutoring and writing assistance, as well
as Supplemental Instruction. All services are free.  

\subsection{Mathematics and Statistics Tutoring Lab} The \href{https://www.american.edu/provost/academic-access/mathstat.cfm}{Mathematics \& Statistics Tutoring Lab} provides free drop-in tutoring and scheduled one-on-one tutoring appointments to American University students for exam review, homework assignments, and understanding concepts. The Lab tutors Mathematics concepts up to Calculus II and Statistics concepts up to Intermediate Statistics. (Drop-in to Don Myers Building Room 103 or Schedule tutoring appointments on WC Online)

\subsection{Writing Center} The \href{https://www.american.edu/provost/academic-access/writing-center.cfm}{Writing Center} offers free, individual coaching sessions to all AU students. In your 45-minute session, a student writing consultant can help you address your assignments, understand the conventions of academic writing, and learn how to revise and edit your own work. (Bender Library – 1st Floor Commons – Schedule tutoring appointments on WC Online – 202-885-2991)

\subsection{Supplemental Instruction} \href{https://www.american.edu/provost/academic-access/supplemental-instruction-homepage.cfm}{Supplemental Instruction (SI)} is a free group tutoring program that supports historically challenging courses in disciplines such as accounting, biology, chemistry and economics. SI Leaders facilitate weekly group review sessions that reiterate course content. In your one or two hour session, an SI Leader can assist with learning course concepts, facilitating group learning, and sharing best strategies for studying and note taking. 

\subsection{Students with Disabilities, ASAC} American University
is committed to making learning and programming as accessible as
possible. Students who wish to request accommodations for a
disability, must notify me with a letter of approved
accommodations from the ASAC. As the process for registering and
requesting accommodations can take some time, and as
accommodations, if approved, are not retroactive, I strongly
encourage students to contact the ASAC as early as possible. For
more information about the process for registering and requesting
disability-related accommodations, contact ASAC.

\subsection{Counseling Center MGC 214, 202-885-3500.} The
\href{http://www.american.edu/counseling}{Counseling Center} helps students make the most of their
university experience, both personally and academically. We offer
individual and group counseling, urgent care, self-help
resources, referrals to private care, as well as programming to
help you gain the skills and insight needed to overcome adversity
and thrive while you are in college. Contact the Counseling
Center to make an appointment in person or by telephone, or visit
the Counseling Center page on the AU website for additional
information.

\subsection{Center for Diversity \& Inclusion MGC 201,
202-885-3651.} \href{http://www.american.edu/ocl/cdi/}{CDI} is dedicated to enhancing LGBTQ,
Multicultural, First Generation, and Women's experiences on
campus and to advance AU's commitment to respecting \& valuing
diversity by serving as a resource and liaison to students,
staff, and faculty on issues of equity through education,
outreach, and advocacy.

\subsection{OASIS: The Office of Advocacy Services for
Interpersonal and Sexual Violence McCabe Hall 123, 202-885-7070,
oasis@american.edu.} OASIS provides free and confidential
advocacy services for students who experience sexual assault,
dating or domestic violence, sexual harassment, and/or stalking.
Please email or call to schedule an appointment with one of the
two victim advocates in OASIS.

\subsection{International Student \& Scholar Services, Batelle
4th Butler Pavilion, Room 410.} \href{https://www.american.edu/ocl/isss/index.cfm}{ISS} has resources to support
academic success and participation in campus life including
academic counseling, support for second language learners,
response to questions about visas, immigration status and
employment and intercultural programs, clubs and other campus
resources.



\end{document}