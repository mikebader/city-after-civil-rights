\documentclass[11pt]{syllabus}

\coursenumber{\ }
\term{\ }
\classtime{\ }
\classroom{\ }
\coursename{The City After Civil Rights}

\author{Michael Bader}
\email{\href{mailto:bader@american.edu}{bader@american.edu}}
\office{\ }

\newcommand{\finalexamperiod}{TBD}
\newcommand{\officehourtimes}{TBD}
\newcommand{\biblocation}{../bib/}

\usepackage[utf8x]{inputenc}
\usepackage[T1]{fontenc}

\begin{document}
\maketitle 

\section{Office Hours}
\officehourtimes

I want to address any thoughts, concerns, or ideas that come up as soon as possible. I can talk in person, over the phone, or via Zoom if you cannot make it onto campus. Please try to make an appointment during the times above. If, however, you cannot meet during those times, \emph{please ask to schedule an alternative appointment time}. I am sure that we can find a mutually convenient time. 

\section{Course Description}
This course examines how American cities have evolved since the United States ratified the radically new vision of race promoted by the Civil Rights Movement in the 1960s. We will study the changing geography of race and class in American cities and their surrounding suburbs and what that evolution has meant for inequality. We will also consider how this shifting geography of race and class affects current debates in metropolitan policies like gentrification and tax policy. We will look to the future to examine what issues might come about in the coming decades and how we might avoid similar problems to those in history. 

We will use the D.C.~area as a venue in which to study many of these phenomena. You will participate by delving into a metropolitan neighborhood, collecting data, stories, and experiences of people in the neighborhood. Together, as a class, we will build a repository of data from all of these different studies of different neighborhoods. 

\begin{objectives}
\item sociologically analyze the historical and contemporary processes that shape the current metropolitan environment, opportunities, and problems;
\item explain the role of metropolitan processes on racial and class inequality in the United States;
\item analyze the demographic composition, organizations, and housing conditions in a particular neighborhood and how that neighborhood fits into the larger metropolitan area; and
\item apply lessons of recent changes to contemporary urban policies and problems confronting U.S. cities
\end{objectives}

\section{Course Policies}

Will be provided in separate handout.

\section{Assignments \& Grading}
\subsection{Assignments}
\begin{description}
\item[Describe a community.] You will work with a group of students over the course of the semester to describe a metropolitan community. You will investigate how the patterns we study in class (e.g., racial composition, class composition, politics, organizations, etc.) affect your assigned community. Your group will choose the \emph{form} of the project, the only requirement is that it must be shareable online. We will discuss this more in the second week of class. 

\item[Synthesis \& further investigation]. You will note in the schedule below several weeks that say "Synthesis \& Further Investigation." These weeks give us the opportunity to identify topics of interest during class discussion that warrant additional thought and more study. \emph{One time} during the semester, your group will lead class discussion. This will include:

\begin{itemize}
    \item summarizing topics of discussion during class
    \item identifying areas where additional reading and discussion would help clarify the topics brought up during class
    \item report on a synthesis of the topics and how it may affect different metropolitan communities
\end{itemize}

\item[Weekly response.] You will submit a response every week that a) summarizes the main argument and supporting evidence of each reading, b) attempt to synthesize the main arguments of the readings to a single statement about the topic covered, c) identify one aspect of the readings that you found confusing, and d) identify at least one aspect of the readings that you found interesting and would like to discuss more. These will be submitted on Canvas.

\item[Class engagement.] Your engagement with the class makes up the final component of grading. This may include participation in classroom discussions. But for those who do not feel as comfortable talking in class, it may also be starting discussions on Canvas or bringing news items to the attention of class. If you have concerns about engaging, please contact me as soon as possible.

\end{description}

\noindent The assignments will be weighted as follows:

\begin{center}
\begin{tabular}{lr}
\textbf{Assignment} & \textbf{Weight}\\\toprule
Describe a community & 50\% (broken into smaller components) \\
Synthesis \& further investigation & 20\%\\
Weekly responses & 20\%\\
Class engagement & 10\%\\\bottomrule
\end{tabular}
\end{center}

\showgrades


\section{Required Texts}
\begin{itemize}
    \item \bibentry{levine_constructing_2021}.
\end{itemize}

\section{Schedule}

\week{Introduction \& Overview}
\begin{readings}
\item National Advisory Commission on Civil Disorders Report (Kerner Commission Report). 1968. Summary. Available online at \url{https://www.ncjrs.gov/pdffiles1/Digitization/8073NCJRS.pdf}.
\item \bibentry{us_department_of_justice_investigation_2015}. Available online at \url{https://www.justice.gov/sites/default/files/opa/press-releases/attachments/2015/03/04/ferguson_police_department_report.pdf}.
\item \bibentry{coates_case_2014}.
\end{readings}

\week{Contemporary Metropolitan Inequality}
\begin{readings}
\item \bibentry{hyra_race_2017} (Selections).
\item \bibentry{lacy_new_2016}.
\item \bibentry{murphy_changing_2015}.
\end{readings}

\week{Race and Place in America}
\begin{readings}
\item \bibentry{logan_global_2010}.
\item \bibentry{kye_detecting_2019}.
\item \bibentry{bader_integration_2021}.
\end{readings}

\week{Are Multiracial Neighborhoods Possible?}
\begin{readings}
\item \bibentry{putnam_e_2007}.
\item \bibentry{abascal_love_2015}.
\item \bibentry{lumley-sapanski_planning_2017}.
\item \bibentry{bader_shared_2021}.
\end{readings}

\week{Synthesis \& Further Investigation of Racial Inequality}

\week{Economic Inequality \emph{Within} Metros}
\begin{readings}
\item \bibentry{ley_liberal_1980}.
\item \bibentry{harvey_managerialism_2001}.
\item \bibentry{rose_rethinking_1984}.
\item \bibentry{freeman_gentrification_2004}.
\end{readings}

\week{Economic Inequality \emph{Across} Metros}
\begin{readings}
\item \bibentry{murray_housing_2018}.
\item \bibentry{whittington_networks_2009}.
\item \bibentry{rodriguez-pose_revenge_2018}.
\end{readings}

\week{Synthesis \& Further Investigation of Economic Inequality}

\week{What is Community?}
\begin{readings}
\item \cite{levine_constructing_2021}, \emph{Constructing Community}.
\end{readings}

\week{Urban Organizations}
\begin{readings}
\item \bibentry{wilson_truly_1987}.
\item \bibentry{small_presence_2006}.
\item \bibentry{small_unanticipated_2009}. Introduction.
\end{readings}

\week{Crime \& Policing}
\begin{readings}
\item \bibentry{sharkey_community_2017}.
\item \bibentry{legewie_contested_2016}.
\item \bibentry{brookings-aei_better_2021}.
\item \bibentry{bader_racial_2020}.
\end{readings}

\week{Synthesis \& Further Investigation of Organizations \& Crime}

\week{Education}
\begin{readings}
\item \bibentry{bader_talk_2019}.
\item \bibentry{lewis-mccoy_inequality_2014}. Selections.
\item \bibentry{bischoff_racial_2018}.
\item \bibentry{bell_serving_1976}.
\end{readings}

\week{Politics}
\begin{readings}
\item \bibentry{spicer_electoral_2018}.
\item \bibentry{figlio_suburbanization_2012}.
\item \bibentry{pape_what_2021}.
\end{readings}

\week{Synthesis \& Further Investigation of Education and Politics}\\[-.25\baselineskip]

\noindent\textbf{Final Presentations: \finalexamperiod}


\bibliographystyle{apalike}
\bibliographystyle{\biblocation asr}
\nobibliography{\biblocation city-after-civil-rights}

\end{document}