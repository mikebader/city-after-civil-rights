\documentclass[11pt]{syllabus}

\coursetitle{The City After Civil Rights}
\coursenumber{AS.230.354}
\author{Prof. Michael Bader}
\email{mbader@jhu.edu}
\office{Mergenthaler 540}


\term{\ }
\classtime{\ }
\classroom{\ }

\newcommand{\finalexamperiod}{TBD}
\newcommand{\officehourtimes}{TBD}
\newcommand{\biblocation}{../bib/}

\usepackage[utf8x]{inputenc}
\usepackage[T1]{fontenc}

\begin{document}
\maketitle 

\section{Office Hours}
\officehourtimes

Office hours offer a time to discuss your own thoughts about the content and to address concerns. I can talk in person, over the phone, or via Zoom if you cannot make it onto campus. Please try to make an appointment during the times above. If, however, you cannot meet during those times, \emph{please ask to schedule an alternative appointment time}. I am sure that we can find a mutually convenient time. 

\section{Course Description}
This course examines how American cities have evolved since the United States ratified the radically new vision of race promoted by the Civil Rights Movement in the 1960s. We will study the changing geography of race and class in American cities and their surrounding suburbs. We will consider how this shifting geography of race and class affects inequality, including its consequences for current debates in metropolitan policies like gentrification and tax policy. We will look to the future to examine what issues might come about in the coming decades and how we might avoid similar problems to those in history. 

We will investigate Baltimore as a venue in which to study many of these phenomena. You will participate by delving into a metropolitan neighborhood, collecting data, stories, and experiences of people in the neighborhood. Together, as a class, we will build a repository of data from all of these different studies of different neighborhoods. 

\begin{objectives}
\item sociologically analyze the historical and contemporary processes that shape the current metropolitan environment, opportunities, and problems;
\item explain the role of metropolitan processes on racial and class inequality in the United States;
\item analyze the demographic composition, organizations, and housing conditions in a particular neighborhood and how that neighborhood fits into the larger metropolitan area; and
\item apply lessons of recent changes to contemporary urban policies and problems confronting U.S. cities
\end{objectives}

/Users/mike/work/teaching/courses/_general-policies.tex

\section{Assignments \& Grading}
\subsection{Assignments}
\begin{description}
\item[Describe a community.] You will work with a group of students over the course of the semester to describe a metropolitan community. You will investigate how the patterns we study in class (e.g., racial composition, class composition, politics, organizations, etc.) affect your assigned community. Your group will choose the \emph{form} of the project, the only requirement is that it must be shareable online. We will discuss this more in the second week of class. 

\item[Weekly response.] You will submit a response every week that a) summarizes the main argument and supporting evidence of each reading, b) attempt to synthesize the main arguments of the readings to a single statement about the topic covered, c) identify one aspect of the readings that you found confusing, and d) identify at least one aspect of the readings that you found interesting and would like to discuss more. These will be submitted on Canvas.

\item[Consider a Question.] Your team will generate a question that interests you based on the material from the first half of the course. You will then investigate that question and generate a list of resources (primarily books and articles) that can help you to answer that question. Your group will supply the list of resources to the class for them to read. 

\item[Class engagement.] Your engagement with the class makes up the final component of grading. This may include participation in classroom discussions. But for those who do not feel as comfortable talking in class, it may also be starting discussions on Canvas or bringing news items to the attention of class. If you have concerns about engaging, please contact me as soon as possible. Class engagement will include coming to office hours \emph{at least once} during the first three weeks of the semester. 

\end{description}

\noindent The assignments will be weighted as follows:

\begin{center}
\begin{tabular}{lr}
\textbf{Assignment} & \textbf{Weight}\\\toprule
Describe a community & 50\% (broken into smaller components) \\
Synthesis \& further investigation & 20\%\\
Weekly responses & 20\%\\
Class engagement & 10\%\\\bottomrule
\end{tabular}
\end{center}

\showgrades


\section{Required Texts}
\begin{itemize}
    \item \bibentry{levine_constructing_2021}.
\end{itemize}

\section{Schedule}

\week{Introduction \& Overview}
\begin{readings}
\item National Advisory Commission on Civil Disorders Report (Kerner Commission Report). 1968. Summary. Available online at \url{https://www.ncjrs.gov/pdffiles1/Digitization/8073NCJRS.pdf}.
\item \bibentry{us_department_of_justice_investigation_2015}. Available online at \url{https://www.justice.gov/sites/default/files/opa/press-releases/attachments/2015/03/04/ferguson_police_department_report.pdf}.
\item \bibentry{coates_case_2014}.
\end{readings}

\week{Contemporary Metropolitan Inequality}
\begin{readings}
\item \bibentry{lacy_new_2016}.
\item \bibentry{hwang_what_2016}.
\item \bibentry{murphy_changing_2015}.
\item \bibentry{goetz_new_2013}. Chapter 1.
\end{readings}

\week{Race and Place in America}
\begin{readings}
\item \bibentry{logan_global_2010}.
\item \bibentry{wright_instability_2020}.
\item \bibentry{bader_integration_2021}.
\end{readings}

\week{Economic Inequality \emph{Within} Metros}
\begin{readings}
\item \bibentry{ley_liberal_1980}.
\item \bibentry{harvey_managerialism_1989}.
\item \bibentry{owens_inequality_2016}.
\end{readings}

\week{Economic Inequality \emph{Across} Metros}
\begin{readings}
\item \bibentry{murray_housing_2018}.
\item \bibentry{whittington_networks_2009}.
\item \bibentry{rodriguez-pose_revenge_2018}.
\item \bibentry{macgillis_fulfillment_2021}. Chapter 9.
\end{readings}

\week{Organizations and Inequality}
\begin{readings}
\item \bibentry{wilson_truly_1987}. Chapters 1 and 2.
\item \bibentry{small_presence_2006}.
\item \bibentry{allard_social_2010}.
\end{readings}

%% QUESTIONS
%% The following modules focus on questions generated by the overview
%% from the first half of the semester. Students should generate their
%% own questions during the first half of the semester. The following 
%% modules can be added and removed as necessary to fill in weeks when 
%% students will not be generating their own questions

\week{What is Community?}
\begin{readings}
\item \cite{levine_constructing_2021}, \emph{Constructing Community}.
\end{readings}

\week{Are Multiracial Neighborhoods Possible?}
\begin{readings}
\item \bibentry{lumley-sapanski_planning_2017}.
\item \bibentry{bader_shared_2021}.
\item \bibentry{lacy_black_2004}.
\end{readings}

\week{What about Multiracial Schools?}
\begin{readings}
\item \bibentry{lewis-mccoy_inequality_2014}. Chapter 4.
\item \bibentry{figlio_suburbanization_2012}.
\item \bibentry{bell_serving_1976}.
\item Please listen to the \emph{Nice White Parents} podcast from the \emph{New York Times}: \url{https://www.nytimes.com/2020/07/23/podcasts/nice-white-parents-serial.html} 
\end{readings}

\week{What Politics Emerge from Place?}
\begin{readings}
\item \bibentry{spicer_electoral_2018}.
\item \bibentry{hyra_backtothecity_2015}.
\item \bibentry{kapos_black_2021}.
\item \bibentry{pape_what_2021}.
\end{readings}

\week{What Problems Arise from Gentrification?}
\begin{readings}
\item \bibentry{demsas_what_2021}.
\item \bibentry{freeman_gentrification_2004}.
\item \bibentry{pattillo_black_2007}. Chapter 2.
\end{readings}

\week{How Do Organizations Help (or Hurt)?}
\begin{readings}
\item \bibentry{dunning_how_2019}.
\item \bibentry{small_unanticipated_2009}. Introduction.
\item \bibentry{sharkey_community_2017}.
\end{readings}


\noindent\textbf{Final Presentations: \finalexamperiod}


\bibliographystyle{apalike}
\bibliographystyle{\biblocation asr}
\nobibliography{\biblocation city-after-civil-rights}

\end{document}